\documentclass{article}
\usepackage{tikz,xcolor}
%\usetikzlibrary{patterns,patterns.meta} % LATEX and plain TEX when using TikZ
\usetikzlibrary{patterns.meta}
\begin{document}
\pgfdeclarepattern{
	name=hatch,
	parameters={\hatchsize,\hatchangle,\hatchlinewidth},
	bottom left={\pgfpoint{-.1pt}{-.1pt}},
	top right={\pgfpoint{\hatchsize+.1pt}{\hatchsize+.1pt}},
	tile size={\pgfpoint{\hatchsize}{\hatchsize}},
	tile transformation={\pgftransformrotate{\hatchangle}},
	code={
		\pgfsetlinewidth{\hatchlinewidth}
		\pgfpathmoveto{\pgfpoint{-.1pt}{-.1pt}}
		\pgfpathlineto{\pgfpoint{\hatchsize+.1pt}{\hatchsize+.1pt}}
		\pgfpathmoveto{\pgfpoint{-.1pt}{\hatchsize+.1pt}}
		\pgfpathlineto{\pgfpoint{\hatchsize+.1pt}{-.1pt}}
		\pgfusepath{stroke}
	}
}
\tikzset{
	hatch size/.store in=\hatchsize,
	hatch angle/.store in=\hatchangle,
	hatch line width/.store in=\hatchlinewidth,
	hatch size=5pt,
	hatch angle=0pt,
	hatch line width=.5pt,
}
\begin{tikzpicture}
\foreach \r in {1,...,4}
    \draw [pattern=hatch, pattern color=red](\r*3,0) rectangle ++(2,2);
\foreach \r in {1,...,4}
    \draw [pattern=hatch, pattern color=green, hatch size=2pt] (\r*3,3) rectangle ++(2,2);
\foreach \r in {1,...,4}
    \draw [pattern=hatch, pattern color=blue, hatch size=10pt, hatch angle=21](\r*3,6) rectangle ++(2,2);
\foreach \r in {1,...,4}
    \draw [pattern=hatch, pattern color=orange, hatch line width=2pt](\r*3,9) rectangle ++(2,2);
\end{tikzpicture}
\end{document}
